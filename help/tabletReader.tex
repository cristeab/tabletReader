\documentclass[journal,12pt]{IEEEtran}
\usepackage[utf8x]{inputenc}

\begin{document}

\title{PDF Reader for Touch-Enabled Devices\\\textrm{(help document)}}
\author{Bogdan Cristea\\\textit{e-mail: cristeab@gmail.com}}

\maketitle

\section{Introduction}

\IEEEPARstart{T}{}abletReader has been designed to be a simple PDF reader for touch-enabled devices (e.g. tablets, hybrid notebooks with touch screen). It can be used also on devices with no touch screen, but in this case a mouse with wheel is needed. TabletReader needs a large enough screen (at least 800x600 pixels). Devices with smaller size screens (e.g. smartphones) may not be suitable for reading PDF documents using A4 or larger page sizes with tabletReader since the displayed text does not adapt to the screen size (as Acrobat Reader does).

Through its simple interface, tabletReader enables to open PDF files found on the local filesystem, to switch between full screen (no toolbar) and normal screen (with toolbar) modes, to go to a specific page, to zoom the displayed document, to show this help, to display information about this software and to exit from the application. At exit, the current PDF file path and name, the current page and the zoom factor are saved so that the next time you launch tabletReader the document is opened at the same page you previously were.

This software has been intended to be used on Linux-based operating systems with a graphical user interface, since there seems to be a lack of PDF readers for touch-enabled devices runnning Linux. However Windows it is also supported and since this software has been writen using Qt framework, it can be ported to other platforms too (provided that poppler library is available).

\newpage

\section{Utilisation}

\subsection{Moving on the same page}
\begin{itemize}
 \item using vertical up and down swipes (you could try this on the current document) or
 \item using the mouse wheel
\end{itemize}

\subsection{Switching between document pages}
\begin{itemize}
 \item using horizontal left and right swipes (you could try this on the current document) or
 \item using the mouse wheel while being at the bottom or at the header of the page
\end{itemize}

\subsection{Toolbar buttons}
\begin{itemize}
 \item \textit{Open}: opens a PDF file found on the local filesystem using the file browser. Vertical up and down swipes (or the mouse wheel) can be used to display the content of the current folder. The first item of the list allows to go up one level, while selecting a folder allows to go down one level. Only files having *.pdf extension are displayed. Please note that a file having this extension is not necessarily a PDF document. If the file cannot be opened, an error message is displayed.
 \item \textit{Full Screen}: the application will occupy the entire screen and the toolbar is no longer displayed. In this mode you need to use the command popup menu available by pressing more than one second the touch screen (see below).
 \item \textit{Go To Page}: displays a numerical keypad allowing to type the desired page number. Note that by pressing OK without typing some page number or if the page number is invalid the numerical key will dissapear and the page will not be changed.
 \item \textit{Zoom}: displays a dialog allowing to select the desired zoom factor.
 \item \textit{Help}: displays this help manual
 \item \textit{About}: displays a dialog with information about this software
 \item \textit{Exit}: exits from the application and saves into a local file the current document path and name, current page and the  zoom factor so that the next time when the application is started the saved settings will be used. Thus you could continue to read the document from the page you closed the application the last time.
\end{itemize}

\subsection{Command popup menu}
The popup menu is available only in full screen mode and it can be displayed by either:
\begin{itemize}
 \item pressing more than one second the touch screen or
 \item clicking and holding pressed for more than one second the right mouse button
\end{itemize}

\textit{Important note}: while in full screen mode the mouse pointer is made invisible, so you need a touch screen in order to use the command popup menu.

The following commands are available:
\begin{itemize}
 \item \textit{Open ...}: same as the \textit{Open} button of the toolbar
\item \textit{Normal Screen}: allows to return to normal screen mode. Alternatively, in order to return to the normal screen mode, one could use the escape key (if a keyboard is available).
 \item \textit{Go To Page ...}: same as the \textit{Go To Page} button of the toolbar
 \item \textit{Zoom ...}: same as the \textit{Zoom} button of the toolbar
 \item \textit{Show Page Number}: displays a small window in the lower left corner of the screen with the current page number and the total number of pages
 \item \textit{Exit}: same as the \textit{Exit} button of the toolbar
\end{itemize}
Thus, the command popup menu allows to handle a PDF document in a so called ``advanced user mode" by providing only the essential commands.

\section{Implementation details}
TabletReader has been written in C++ using Qt application framework, QML for the graphical user interface and poppler library for handling pdf documents. The application is build around a circular buffer holding three pages which allows to display the PDF document pages using the \textrm{SlidingStackedWidget} class. In general, the circular buffer will hold the currently displayed page, the previous page and the next one so that when the user switches between adjacent pages this is done as fast as possible. 

When loading a new document, only the current page is loaded in the main thread, while the previous and the next pages are loaded in a secondary thread (\textrm{Worker} class). Similarly, when changing between adjacent pages, the page from the circular buffer is displayed and then a new page is loaded in the background by the secondary thread so that, with respect to the currently displayed page, the previous and the next page are always ready to be displayed in the circular buffer.

The current application settings (current document, current page, zoom factor) are stored using the \textrm{QSettings} class in a portable format between operating systems. However, in order to save the current settings the application must be closed using the \textit{Exit} command. If the device is shutdown while using the application the current settings are not saved because there is no way to catch the termination signal. If no settings are available, when the application is started this help document is displayed.

\section{License}
 Copyright \copyright 2011, Bogdan Cristea
 
 This program is free software; you can redistribute it and/or modify  it under the terms of the GNU General Public License as published by  the Free Software Foundation; either version 2, or (at your option)  any later version.
 
 This program is distributed in the hope that it will be useful,  but WITHOUT ANY WARRANTY; without even the implied warranty of
 MERCHANTABILITY or FITNESS FOR A PARTICULAR PURPOSE.  See the  GNU General Public License for more details.
 
 You should have received a copy of the GNU General Public License along with this program; if not, write to the Free Software
 Foundation, Inc., 51 Franklin Street - Fifth Floor, Boston, MA 02110-1301, USA.

\end{document}
